
\let\negmedspace\undefined
\let\negthickspace\undefined
\documentclass[journal,12pt,twocolumn]{IEEEtran}
       \def\inputGnumericTable{}                                 %%
\usepackage{cite}
\usepackage{amsmath,amssymb,amsfonts,amsthm}
\usepackage{algorithmic}
\usepackage{graphicx}
\usepackage{textcomp}
\usepackage{xcolor}
\usepackage{txfonts}
\usepackage{listings}
\usepackage{enumitem}
\usepackage{mathtools}
\usepackage{gensymb}
\usepackage[breaklinks=true]{hyperref}
\usepackage{tkz-euclide} % loads  TikZ and tkz-base
\usepackage{listings}
\usepackage{gvv}
%
%\usepackage{setspace}
%\usepackage{gensymb}
%\doublespacing
%\singlespacing

%\usepackage{graphicx}
%\usepackage{amssymb}
%\usepackage{relsize}
%\usepackage[cmex10]{amsmath}
%\usepackage{amsthm}
%\interdisplaylinepenalty=2500
%\savesymbol{iint}
%\usepackage{txfonts}
%\restoresymbol{TXF}{iint}
%\usepackage{wasysym}
%\usepackage{amsthm}
%\usepackage{iithtlc}
%\usepackage{mathrsfs}
%\usepackage{txfonts}
%\usepackage{stfloats}
%\usepackage{bm}
%\usepackage{cite}
%\usepackage{cases}
%\usepackage{subfig}
%\usepackage{xtab}
%\usepackage{longtable}
%\usepackage{multirow}
%\usepackage{algorithm}
%\usepackage{algpseudocode}
%\usepackage{enumitem}
%\usepackage{mathtools}
%\usepackage{tikz}
%\usepackage{circuitikz}
%\usepackage{verbatim}
%\usepackage{tfrupee}
%\usepackage{stmaryrd}
%\usetkzobj{all}
    \usepackage{color}                                            %%
    \usepackage{array}                                            %%
    \usepackage{longtable}                                        %%
    \usepackage{calc}                                             %%
    \usepackage{multirow}                                         %%
    \usepackage{hhline}                                           %%
    \usepackage{ifthen}                                           %%
 %optionally (for landscape tables embedded in another document): %%
    \usepackage{lscape}     
%\usepackage{multicol}
%\usepackage{chngcntr}
%\usepackage{enumerate}

%\usepackage{wasysym}
%\documentclass[conference]{IEEEtran}
%\IEEEoverridecommandlockouts
% The preceding line is only needed to identify funding in the first footnote. If that is unneeded, please comment it out.

\newtheorem{theorem}{Theorem}[section]
\newtheorem{problem}{Problem}
\newtheorem{proposition}{Proposition}[section]
\newtheorem{lemma}{Lemma}[section]
\newtheorem{corollary}[theorem]{Corollary}
\newtheorem{example}{Example}[section]
\newtheorem{definition}[problem]{Definition}
%\newtheorem{thm}{Theorem}[section] 
%\newtheorem{defn}[thm]{Definition}
%\newtheorem{algorithm}{Algorithm}[section]
%\newtheorem{cor}{Corollary}
\newcommand{\BEQA}{\begin{eqnarray}}
\newcommand{\EEQA}{\end{eqnarray}}
\newcommand{\define}{\stackrel{\triangle}{=}}
\theoremstyle{remark}
\newtheorem{rem}{Remark}

%\bibliographystyle{ieeetr}
\begin{document}
Consider a triangle with vertices
\begin{align}
\vec {A} = \myvec {-4\\-3\\}\\
\vec {B} = \myvec {-6\\1\\}\\
\vec {C} = \myvec {-5\\-5\\}
\end{align}
\begin{table}[!ht]
        \centering
        \caption{Trianle}
	\label{table 1}	
	\input{/home/yogitha/triangle/tables/t1.tex}
\end{table}
\begin{figure}[h!]
\centering
\includegraphics[width=0.8\linewidth] {/home/yogitha/triangle/figs/Fig1.png}
\caption{Sides}
\label{Fig1}
\end{figure}
\begin{table}[!ht]
        \centering
        \caption{Medians}
	\label{table 2}
	\input{/home/yogitha/triangle/tables/t2.tex}
\end{table}
\begin{figure}[h!]
\centering
\includegraphics[width=0.8\linewidth] {/home/yogitha/triangle/figs/Fig2.png}
\caption{Medians}
\label{fig2}
\end{figure}
\begin{table}[!ht]
        \centering
	\caption{Medians}
	\label{table 3}
	\input{tables/t3.tex}
\end{table}
\begin{figure}[h!]
\centering
\includegraphics[width=0.8\linewidth] {/home/yogitha/triangle/figs/Fig3.png}
\caption{Altitudes}
\label{fig3}
\end{figure}
\begin{table}[!ht]
        \centering
        \caption{Perpendicular bisectors}
	\label{Table 4}	
	\input{/home/yogitha/triangle/tables/t4.tex}
\end{table}
\begin{figure}[h!]
\centering
\includegraphics[width=0.8\linewidth] {/home/yogitha/triangle/figs/Fig4.png}
\caption{Perpendicular bisectors}
\label{fig4}
\end{figure}
\begin{table}[!ht]
        \centering
        \caption{Angular bisectors}
	\label{table 5}	
	\input{/home/yogitha/triangle/tables/t5.tex}
\end{table}
\begin{figure}[h!]
\centering
\includegraphics[width=0.8\linewidth] {/home/yogitha/triangle/figs/Fig5.png}
\caption{Angular bisectors}
\label{fig5}
\end{figure}
\end{document}
